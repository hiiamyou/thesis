% Copyright (c) 2008-2009 solvethis
% Copyright (c) 2010-2016,2018-2019,2022 Casper Ti. Vector
% Copyright (c) 2021 Kurapica
% Copyright (c) 2022 iofu728
% Overleaf version.

%*********************************************************************
% iofu728-pkuthss: 北京大学研究生学位论文模板
% 2022/07/16 v1.1.0
%
% 重要提示:
%   1. 当前overleaf版符合2022研究生学位论文要求,可通过图书馆审核
%   2. 当前版本基于pkuthss v1.9.2
%   3. 请使用UTF-8编码,XeLaTeX方式编译
%   4. 请仔细阅读用户文档
%   5. 修改、使用、发布本文档类请务必遵循LaTeX Project Public License和知识共享4.0
%   6. 如有疑问github/iofu728/pkuthss上提问或联系作者@iofu728
%*********************************************************************

%   引入文档类 pkuthss
\documentclass[fontset=fandol,ugly,language=chinese]{pkuthss}
  % 学位论文模式  ugly    (默认打开,请保留)
  % 盲审模式     blind   (默认关闭)
  % 语言        language (默认chinese | english)
  % 字体库       fontset
  %   auto | windows | windows@overleaf | mac | fandol | ubuntu | none
  % windows*, mac为商业字体,如需使用请遵循相应版权协议(默认下overleaf中不可用)
  % fandol与windows效果相近,但字符库偏少,推荐使用(默认);
  % ubuntu字体效果偏差较大; 设为none时需自行配置字体集;

\usepackage[backend=biber,style=gb7714-2015]{biblatex}
  % 参考文献遵循GB/T 7714-2015标准,使用biblatex-gb7714-2015 宏包。
  % 此处使用顺序编码制,如使用著者-出版年制则更改为b7714-2015ay。

% 示例文档用包和设定,该段均可移除.
\usepackage{enumitem,fancyvrb}
\usepackage{booktabs,multirow,longtable,makecell} % 表格相关
\RecustomVerbatimEnvironment{Verbatim}{Verbatim}{frame = single, tabsize = 4, fontsize=\footnotesize}
\renewcommand{\v}[1]{\boldsymbol{#1}}
\newcommand\pkg[1]{\textsf{#1}}

% 参考文献边距字体
\setlength{\bibitemsep}{3bp}
\renewcommand*{\bibfont}{\zihao{5}\linespread{1.27}\selectfont}

%   pkuthss.cls 中定义的 newcommand —— pkuthss,填写信息
\pkuthssinfo{
    cthesisname = {硕士学位论文},
    thesiscover = {硕士研究生学位论文},
    ethesisname = {Master Thesis},
    ctitle = {一个微服务监测系统中Trace后端模块的设计与实现},
    etitle = {Design and Implementation of a Microservice Monitoring System Trace Back End Module},
    cauthor = {周天寅}, eauthor = {Tianyin Zhou},
    studentid = {2001210722},
    % 具体时间以教务为准,初稿3月,送审4月,答辩5月,最终6月。
    date = {\zhdigits{2023}\ \ 年\ \ \zhnumber{6}\ \ 月}, % June, 2022
    school = {软件与微电子学院},
    cmajor = {通信工程}, emajor = {Communication Engineering},
    direction = {智能系统设计},
    mentorlines = {1}, % 导师个数
    % 副教授 A.P. 讲师 Lec.
    cmentor = {林金龙\ \ 教授}, ementor = {Prof.\ Jinlong Lin},
    ckeywords = {A,B,C,D},
    ekeywords = {A,B,C,D},
    % 盲审模式参数, 需在documentclass增加blind
    blindid = {XXXXXXXXX}, discipline = {XXXX}
}

%   加入参考文献
\addbibresource{ref.bib}


%   !!!用于去除摘要前后等位置的空白页
\let\cleardoublepage\clearpage


%   在 begin{document} 之前为导言区,或理解为模板定义 
\begin{document}

    %   frontmatter 在 \maketitle 前,改页码为罗马数字,并使得节不计数,用来排版书籍前言
    \frontmatter
    %   empty 不设页眉页脚
    \pagestyle{empty}
    %   制作首页
    \maketitle
    %   用 \clearpage 命令清空浮动体队列,并开始新的一页
    \cleardoublepage
    %   用 \cleardoublepage 命令清空浮动体队列,并在偶数页开始新的一页

    %   版权声明,需替换门户版权声明pdf
    \include{chap/copy}
    \cleardoublepage
    
    %   摘要,中英文
    \pagestyle{plain}
    \setcounter{page}{0}
    \pagenumbering{Roman}
    \begin{cabstract}
    实验室 事实上对方中文摘要部分...
\end{cabstract}

\begin{eabstract}
    英文摘要部分...
\end{eabstract}

% vim:ts=4:sw=4


    %   加入目录
    \tableofcontents
    
    % 如有需要使用表格索引、插图索引
    % \listoftables
    % \listoffigures
    
    % 如有需要使用主要符号对照表
    % \include{chap/deno}

    %   mainmatter 在第一章之前,以阿拉伯数字开始页码计数。用来排版书籍主体
    \mainmatter
    \chapter{引言}
\label{chap:introduction}

本章

\section{背景}
\label{sec:background}

pkuthssinfo相

    \include{chap/chap2}
    \include{chap/chap3}
    \include{chap/chap4}
    \include{chap/chap5}
    \include{chap/chap6}

    %   appendix 用来标识附录内容,改页码为字母
    \appendix
    
    \printbibliography[heading = bibintoc]
    % 如有需要使用研究生成果页
    \include{chap/ach}

    \backmatter
    \include{chap/ack}
    % 需替换门户原创页pdf/扫描pdf
    \include{chap/origin}

\end{document}

% vim:ts=4:sw=4